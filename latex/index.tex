This is a simple project that uses Arduino and a Linux-\/based Single-\/\+Board computer to monitor ambient variables that influence your sleep quality. With this information the system can send alarms with insight about your sleeping ambient in order to improve it.

\subsection*{Table of Contents}


\begin{DoxyItemize}
\item \href{#getting-started}{\tt Getting Started}
\item \href{#prequsites}{\tt Prequisites}
\item \href{#usage}{\tt Usage}
\begin{DoxyItemize}
\item \href{#home-page}{\tt Main page}
\item \href{#reports}{\tt Reports}
\item \href{#configure}{\tt Configure}
\end{DoxyItemize}
\end{DoxyItemize}

\subsection*{Getting Started}

This is a list of things you will need in order to use this code
\begin{DoxyItemize}
\item An Arduino Board (Uno or Nano will do just fine)
\begin{DoxyItemize}
\item A D\+H\+T11 sensor (Temperature and Humidity)
\item An L\+DR
\item A 220 Ohms resistor. You can change this but you would need to modify some Arduino source code.
\item Implementing this circuit 
\end{DoxyItemize}
\item A Linux-\/\+Based Single-\/\+Board Computer (We recommend a Raspberry Pi and Raspbian) \subsection*{Prequsites}
\end{DoxyItemize}

First of all, you need to install the software in the Arduino. You can do this in any way you prefer (I\+DE or console).

The code that must be installed is {\ttfamily unifiend\+\_\+sensors\+\_\+sm.\+ino}. Remember to have the circuit configuration as shown before. Otherwise you will need to change the code.

After that you will need to connect via U\+SB your Arduino to your S\+BC.
\begin{DoxyItemize}
\item First you must install S\+Q\+Lite3 in your computer. To do so run the following commands, that will install S\+Q\+Lite3 on your S\+BC.
\begin{DoxyItemize}
\item {\ttfamily sudo apt-\/get install sqlite3}
\item {\ttfamily sudo apt-\/get install sqlitebrowser}
\end{DoxyItemize}
\item Then you will need to run the requirements.\+txt file. To do that remember to set a virtual enviorment for this app. Then, inside the venv just run {\ttfamily pip install -\/-\/user -\/-\/requirement requirements.\+txt}. This will leave all the required libraries for the project to work readily installed in your virtual environment.
\item After that you must create yourself a \href{https://accounts.google.com/signup/v2/webcreateaccount?flowName=GlifWebSignIn&flowEntry=SignUp}{\tt gmail account} and \href{https://myaccount.google.com/lesssecureapps}{\tt set it} up to work for app S\+M\+TP.
\begin{DoxyItemize}
\item Then you must configure your account in the {\ttfamily email\+\_\+sender.\+py} module. For the {\ttfamily U\+S\+E\+R\+N\+A\+ME} constant write the email of your newly created account. Now open the Python Console and configure \href{https://pypi.org/project/keyring/}{\tt keyring} to use it in the app.
\end{DoxyItemize}
\item After that you must run two different Python scripts in your S\+BC, in different consoles. Leave the second one running\+:
\begin{DoxyItemize}
\item {\ttfamily create\+\_\+database.\+py}
\item {\ttfamily serial\+\_\+reader.\+py}
\end{DoxyItemize}

That\textquotesingle{}s it for the prequisites part!
\end{DoxyItemize}

\subsection*{Usage}

Now to the interesting part\+: how do you use this?

First, we must pay attention to the IP direction and the port on which the app is running. This will route us to the project from any local network connected device.

You must learn, by the tool of your choice (we recommend net-\/tools), your IP direction in your local network.

After that you must run two commands (from the project venv)\+:

{\ttfamily export F\+L\+A\+S\+K\+\_\+\+A\+PP=hello.\+py}

{\ttfamily flask run -\/-\/host=0.\+0.\+0.\+0}

This will allow you to use the project in your local network\+: you will need to access, from any local network connected device, the IP $<$S\+B\+C-\/\+IP$>$\+:5000, which will show you the project working.

\subsubsection*{Home Page}



The first thing you can do is see the actual state of the environment. The colour of the value representing each value tells you how ideal is that value for sleeping
\begin{DoxyItemize}
\item Green is ideal
\item Yellow is not so bad
\item Red is not a recommendable value
\end{DoxyItemize}



You can also see three tables that show you a history of all three variables. They get generated when you ask for the last N registers.

\subsubsection*{Reports}

You can also create your own reports based on dates from a time period you select.



After that, on another page, you will receive a report which you can later download if you wish so.



\subsubsection*{Configure}

In order to receive emails with your report after every sleep session you must register the sleeping ranges for every day and the email you want to have the email with the report sent.



Five minutes after your sleep session is over you will receive an email with the report



\subsection*{Contributors}

This project exists thanks to all the people who contribute.
\begin{DoxyItemize}
\item \href{https://github.com/FranzSL}{\tt Franz Soto Leal}
\item \href{https://github.com/ropa1998}{\tt Rodrigo Pazos}
\end{DoxyItemize}

\subsection*{License}

\mbox{[}M\+IT\mbox{]}(L\+I\+C\+E\+N\+SE) 